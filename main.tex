%%%% Opcje
%% wmii --- Wydział Matematyki i informatyki
%%
%% Kierunek:
%%      inf --- Informatyka
%%
%% Poziom studiów (praca):
%%      mgr --- magisterska




\documentclass[wmii, inf, mgr]{uwmthesis}

\usepackage[utf8]{inputenc}
\usepackage[MeX]{polski}
\usepackage{graphicx}
\usepackage{url}
\usepackage{float}

\usepackage{listings}
\lstset{language=csh,
  showspaces=false,
  showtabs=false,
  breaklines=true,
  showstringspaces=false,
  breakatwhitespace=true,
  escapeinside={(*@}{@*)},
  basicstyle=\ttfamily,
  columns=fullflexible
}%% dla listingów kodu



\date{2022}
\title{Zastosowanie data science do analizowania preferencji zakupowych użytkowników oraz sugerowania im potencjalnych towarów i usług.}
\author{Mateusz Śliwiński}
\etitle{Using data science to analyze users' shopping preferences and suggest potential goods and services.}
\wykonanaw{Katedra Metod Matematycznych Informatyki}
\ewykonanaw{the Chair of Mathematical Methods of Computer Science}

\podkierunkiem{dr. Andrzeja Jankowskiego}
\epodkierunkiem{Andrzej Jankowski, PhD}

\begin{document}

\maketitle





\begin{streszczenie}
szablon streszczenia

\end{streszczenie}

\begin{abstract}
dfsf 

\end{abstract}


\tableofcontents

\chapter*{Wstęp}

sdfsdfs

\chapter{Wymagania aplikacji}

sgsgfsdg

\chapter{Użyte technologie}

fgxf

\chapter{Źródła danych }
dasdsa

\chapter{Wizualizacja danych }
dasdsa

\chapter{Trenowanie modeli }
dasdsa


\chapter{Opis wykorzystanych algorytmów }
(tutaj jeszcze nie wiem jakich użyję więc wrzucam jak najwięcej, żeby mieć potem w czym wybierać)
\section{Mierzenie dokładności algorytmów decyzyjnych}

tutaj coś o błędach

dfghdfgh


\section{Sieć neuronowa}

dfghdfgh


\section{Drzewo decyzyjne – regresyjne}

dfghdfgh

\section{Regresyjny las losowy (ang. Regression forest)}

fghfgh

\section{Drzewo decyzyjne wzmocnione}

ghjghj


hjkgk




\chapter{Implementacja i omówienie kodu}


\section{Algorytmy}
fudtuy


\section{Łączenie algorytmów}
fudtuy



\chapter{Sprawdzanie efektywności uczenie zaespołowego- AB testy}

cgjh

\chapter{Podsumowanie}

gh


\listoffigures
dsa

\listoftables
sdadsa

\chapter{Spis algorytmów}
sdadsa

\chapter{Indeks stosowanych oznaczeń}
sdadsa

\chapter{Używane źródła danych i dokumentów z Internetu}
\textit{Artificial neural network}
\url{https://en.wikipedia.org/wiki/Artificial_neural_network}, dostęp online: 20.01.2022.


\thebibliography{1}
\bibitem{bib1}
\textit{Artificial neural network}
 \url{https://en.wikipedia.org/wiki/Artificial_neural_network}, dostęp online: 20.01.2022.


\end{document}
